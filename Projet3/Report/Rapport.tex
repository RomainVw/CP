\documentclass[a4paper ,12pt,french]{article}
% Packages usuels
%\usepackage{etex} % pour circuitikz
%\usepackage{tikz}
%\usepackage{circuitikz} % pour les circuits électriques

\usepackage[utf8]{inputenc}
%\usepackage[margin=1.55cm]{geometry}
\usepackage[bottom=2cm , left=2.5cm ,right=2.5cm, top=2cm]{geometry}
\usepackage[T1]{fontenc}
\usepackage{url}
%\usepackage{color}
\usepackage{lmodern}
\usepackage[french]{babel}
\usepackage[indentfirst]{titlesec}
\usepackage[dvips]{graphicx}
\usepackage{eurosym}
\usepackage{amsmath}
\usepackage{amsfonts}
\usepackage{amssymb}
\usepackage{makeidx}
\usepackage{array}
\usepackage{colortbl}
\usepackage[table,dvipsnames,svgnames]{xcolor}
\usepackage{xspace}
\usepackage{fancybox}
\usepackage{textcomp}
\usepackage{listings}


\usepackage{hyperref}
\usepackage{setspace}
\usepackage{fancyhdr}
\usepackage{graphicx}

%\usepackage[margin=0.75in]{geometry}
\usepackage[version=3]{mhchem}
%\usepackage{chemist}
\usepackage{multicol}
\usepackage{float}
\usepackage{wrapfig} %écrire txt et image côte à côte
\usepackage[rightcaption]{sidecap}
\usepackage{amsthm}
\usepackage[squaren, Gray, cdot]{SIunits}
\usepackage[absolute]{textpos}%positionnement de cadres
\usepackage[final]{pdfpages} %traitement des pdf
\usepackage{subfigure}
%\usepackage[framed,numbered,autolinebreaks,useliterate]{mcode}%pour traiter le code matlab
%\usepackage{setspace}% pour les interlignes
%\onehalfspacing %interligne 1.5
%\doublespacing %interligne 2
%\renewcommand{\baselinestretch}{1.5}  %interligne défini
%\usepackage{vmargin}% pour les marges
%\setmarginsrb{2.5}{2.5}{2.5}{2.5}{}{}{}{} % marges de 2.5 cm 
%\addto\captionsfrench{\def\tablename{Tableau}} % pour avoir TABLEAU et pas TABLE dans la légende des tableaux..
%\setlength{\parskip}{1cm}   %espacement fixe entre chaque paragraphe
\setlength{\parindent}{1cm}  %modifie la valeur de l'alinéas
%\addtolength{\voffset}{-1.5cm} % (diminue la marge du haut)
\addtolength{\textheight}{-2cm} % (augmente la longueur du texte)
%\addtolength{\hoffset}{-1cm} (diminue la marge de gauche)
%\addtolength{\textwidth}{2cm}  (augmente la largeur du texte)
%\addtocounter{secnumdepth}{1}  si jamais on veut utiliser \subsubsubsecion
\usepackage[hang,center,bf]{caption} %pour les légendes
\setlength{\captionmargin}{30pt}
\usepackage[hang,flushmargin]{footmisc} %à mettre avec ENGLISH dans babel pour avoir les notes de bas de page à gauche et non indentées
\usepackage[nonumberlist,style=altlist,toc]{glossaries} % Pour faire un glossaire
\makeglossaries
%\addto\captionsfrench{\renewcommand*{\glossaryname}{Glossary}}
\usepackage{wasysym}
\usepackage[square, numbers, comma, sort&compress]{natbib} % Use the natbib reference package - read up on this to edit the reference style; if you want text (e.g. Smith et al., 2012) for the in-text references (instead of numbers), remove 'numbers' 
%\hypersetup{urlcolor=blue, colorlinks=true} % Colors hyperlinks in blue - change to black if annoying
\title{Project 2 - Constraint Programming } % Defines the thesis title - don't touch this
%-------------------------------------------------------------------------------------------------------------------------------------------------------------




\begin{document}

\definecolor{dkgreen}{rgb}{0,0.6,0}
\definecolor{gray}{rgb}{0.5,0.5,0.5}
\definecolor{mauve}{rgb}{0.58,0,0.82}

\lstset{ %
  language=c,                				% the language of the code
  basicstyle=\footnotesize,           	% the size of the fonts that are used for the code
  numbers=left,                   			% where to put the line-numbers
  numberstyle=\tiny\color{gray},  	% the style that is used for the line-numbers
  stepnumber=1,                   			% the step between two line-numbers. If it's 1, each line 
                                  						% will be numbered
  numbersep=5pt,                  			% how far the line-numbers are from the code
  backgroundcolor=\color{white},   % choose the background color. You must add \usepackage{color}
  showspaces=false,               % show spaces adding particular underscores
  showstringspaces=false,         % underline spaces within strings
  showtabs=false,                 % show tabs within strings adding particular underscores
  frame=single,                   % adds a frame around the code
  rulecolor=\color{black},        % if not set, the frame-color may be changed on line-breaks within not-black text (e.g. commens (green here))
  tabsize=4,                      % sets default tabsize to 2 spaces
  captionpos=b,                   % sets the caption-position to bottom
  breaklines=true,                % sets automatic line breaking
  breakatwhitespace=false,        % sets if automatic breaks should only happen at whitespace
  title=\lstname,                   % show the filename of files included with \lstinputlisting;
                                  % also try caption instead of title
  keywordstyle=\color{blue},          % keyword style
  commentstyle=\color{dkgreen},       % comment style
  stringstyle=\color{mauve},         % string literal style
  escapeinside={\%*}{*)},            % if you want to add LaTeX within your code
  morekeywords={*,...}               % if you want to add more keywords to the set
}

\floatstyle{plain}
%\newfloat{graphique}{!hb}{lgr}[chapter]
\floatname{graphique}{Graph}

%\setstretch{1.1} % Line spacing of 1.3

% Define the page headers using the FancyHdr package and set up for one-sided printing
\fancyhead{} % Clears all page headers and footers
\rhead{\thepage} % Sets the right side header to show the page number
\lhead{} % Clears the left side page header

\pagestyle{fancy} % Finally, use the "fancy" page style to implement the FancyHdr headers

\newcommand{\HRule}{\rule{\linewidth}{0.5mm}} % New command to make the lines in the title page

\begin{titlepage}
\pagestyle{fancy} % Finally, use the "fancy" page style to implement the FancyHdr headers

\begin{tabular}{cc}
\begin{minipage}{0.5\textwidth}
\begin{flushleft}
\includegraphics[scale=0.1]{./logoingisbleu.jpg} % University/department logo - uncomment to place it
\end{flushleft}
\end{minipage}
 & 
 \begin{minipage}{0.43\textwidth}
\begin{flushright}
\includegraphics[scale=0.5]{./epl.jpg} % University/department logo - uncomment to place it
\end{flushright}
\end{minipage}
\end{tabular} 



\begin{center}
\vspace{100 px}
\textsc{\LARGE Catholic University of Louvain}\\[1cm] % University name
\textsc{\Large Project 3 : Search}\\[0.5cm] % Thesis type
 
\HRule \\[0.4cm] % Horizontal line
{\huge \bfseries LINGI2365 - Constraint Programming}\\[0.4cm] % Thesis title
\HRule \\[1.5cm] % Horizontal line
 

\begin{tabular}{cc}
\begin{minipage}{0.5\textwidth}
\begin{flushleft} \large
\emph{Auteurs:}\\
{Vanwelde Romain (3143-10-00)\\
Crochelet Martin (2236-10-00)\\ \ \\
Groupe 7} 
\end{flushleft}
\end{minipage} & \begin{minipage}{0.46\textwidth}
\centering
\begin{flushright} \large
\emph{Superviseurs:} \\
{Pr. Yves Deville\\
François Aubry
}
\end{flushright}
\end{minipage}\\[3cm] \\ 
\end{tabular} 

 
%\large \textit{A thesis submitted in fulfilment of the requirements\\ for the degree of \degreename}\\[0.3cm] % University requirement text
%\textit{in the}\\[0.4cm]
%\groupname\\\deptname\\[2cm] % Research group name and department name

 \begin{center}
{\large \today }\\[4cm] % Date 
 \end{center}


\vfill
\end{center}

\end{titlepage}

\lhead{\emph{Table of Contents}} % Set the left side page header to "Contents"
\tableofcontents % Write out the Table of Contents

\thispagestyle{fancy}

\pagebreak
\setcounter{page}{1}
\pagestyle{fancy} % Finally, use the "fancy" page style to implement the FancyHdr headers

\section{The Brussels airport problem}

\subsection{Explain the given model}
\subsection{Design 2 different variable and/or value ordering heuristics for this problem.}
\subsection{Which criteria are meaningful for comparing different search strategies?}
\subsection{Based on your criteria, compare your heuristics with the labelFF heuristic by testing them on the instance on iCampus.}
\subsection{Consider the following strategy. \dots Give an example with three planes where this strategy is wrong}

\section{The Knapsack Problem}

\subsection{A Branch \& Bound approach}

\subsubsection{Model the knapsack problem as Constraint Optimization}
\subsubsection{Describe your model in the report.}
\subsubsection{Design 3 different heuristics for variable selection.}
\subsubsection{Test your heuristics and the labelFF heuristic on the knapsack-A instances.}
\subsubsection{Present and discuss the results in your report}


%Solution&21&21&21&21&26&34&43&44&45&47\\


time :\\
\begin{tabular}{|r||r|r|r|r|r|r|r|r|r|r|r|r|r|r|r|r|r|r|r|r|r|r|r|r|r|}
\hline
&H1&H2&H3&H4&H5&H6&H7&H8&H9&H10\\
\hline
\hline
s1&76&37&21&50&73&95&219&318&1098&2189\\
\hline
s2&15&10&9&14&20&27&63&124&213&349\\
\hline
s3&29&21&22&28&34&63&145&166&899&1476\\
\hline
ff&17&15&10&18&26&45&98&113&672&1034\\
\hline
\end{tabular}\\

\#choices\\
\begin{tabular}{|r||r|r|r|r|r|r|r|r|r|r|r|r|r|r|r|r|r|r|r|r|r|r|r|r|r|}
\hline
&H1&H2&H3&H4&H5&H6&H7&H8&H9&H10\\
\hline
\hline
s1&2151&1068&609&1488&2058&2668&6177&10206&36020&73265\\
\hline
s2&489&343&329&569&697&891&2309&3986&8218&12734\\
\hline
s3&1219&678&736&858&1016&2103&4877&5850&29973&49940\\
\hline
s4&538&468&339&617&860&1590&3756&4431&26737&41081\\
\hline

\end{tabular}\\


\#fail\\
\begin{tabular}{|r||r|r|r|r|r|r|r|r|r|r|r|r|r|r|r|r|r|r|r|r|r|r|r|r|r|}
\hline
&H1&H2&H3&H4&H5&H6&H7&H8&H9&H10\\
\hline
\hline
s1&4266&2098&1140&2940&3975&5298&12193&20306&71907&146472\\
\hline
s2&796&507&473&790&1186&1451&3792&7176&13027&21112\\
\hline
s3&1731&1120&1045&1454&1852&3667&8608&10247&55578&90988\\
\hline
s4&722&574&437&847&1149&1966&4781&5723&33206&54804\\
\hline
\end{tabular}\\


\#propag:\\
\begin{tabular}{|r||r|r|r|r|r|r|r|r|r|r|r|r|r|r|r|r|r|r|r|r|r|r|r|r|r|}
\hline
&H1&H2&H3&H4&H5&H6&H7&H8&H9&H10\\
\hline
\hline
s1&25586&12644&7115&16302&25829&36401&81296&129069&427978&819394\\
\hline
s2&3634&2197&1840&2451&4878&6845&15387&34646&50685&91129\\
\hline
s3&6250&5299&3798&6189&8609&17699&41002&48293&280535&391867\\
\hline
s4&6306&5371&3867&6210&8680&17772&41030&48393&280603&391948\\
\hline
\end{tabular}


\subsection{Optimization over iterations}

\subsubsection{Model the knapsack problem as a Constraint Satisfaction Problem.}
\subsubsection{In order to implement the optimization over iterations \dots}

\subsubsection{Which of these points (i., ii., iii.) do you need to execute on which events?}
\subsubsection{How do you modify the value of ub to be sure to find the optimal solution?}
\subsubsection{Can you explain why we initialize ub with an upper bound instead of any other value?}
\subsubsection{Experiment this program on the instances knapsack-A, -B. }
\subsubsection{Present and discuss the results in your report.}

\subsection{Optimization via divide and conquer}

\subsubsection{In order to implement the optimization via divide and conquer you will have to \dots}

\subsubsection{Which of these points (i., ii., iii., iv.) do you need to execute on which events?}
\subsubsection{Experiment this version on the instances knapsack-A,-B,-C}
\subsubsection{Present and discuss the results in your report.}

\end{document}