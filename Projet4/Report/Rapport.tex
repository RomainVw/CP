\documentclass[a4paper ,12pt,french]{article}
% Packages usuels
%\usepackage{etex} % pour circuitikz
%\usepackage{tikz}
%\usepackage{circuitikz} % pour les circuits électriques

\usepackage[utf8]{inputenc}
%\usepackage[margin=1.55cm]{geometry}
\usepackage[bottom=2cm , left=2.5cm ,right=2.5cm, top=2cm]{geometry}
\usepackage[T1]{fontenc}
\usepackage{url}
%\usepackage{color}
\usepackage{lmodern}
\usepackage[french]{babel}
\usepackage[indentfirst]{titlesec}
\usepackage[dvips]{graphicx}
\usepackage{eurosym}
\usepackage{amsmath}
\usepackage{amsfonts}
\usepackage{amssymb}
\usepackage{makeidx}
\usepackage{array}
\usepackage{colortbl}
\usepackage[table,dvipsnames,svgnames]{xcolor}
\usepackage{xspace}
\usepackage{fancybox}
\usepackage{textcomp}
\usepackage{listings}


\usepackage{hyperref}
\usepackage{setspace}
\usepackage{fancyhdr}
\usepackage{graphicx}

%\usepackage[margin=0.75in]{geometry}
\usepackage[version=3]{mhchem}
%\usepackage{chemist}
\usepackage{multicol}
\usepackage{float}
\usepackage{wrapfig} %écrire txt et image côte à côte
\usepackage[rightcaption]{sidecap}
\usepackage{amsthm}
\usepackage[squaren, Gray, cdot]{SIunits}
\usepackage[absolute]{textpos}%positionnement de cadres
\usepackage[final]{pdfpages} %traitement des pdf
\usepackage{subfigure}
%\usepackage[framed,numbered,autolinebreaks,useliterate]{mcode}%pour traiter le code matlab
%\usepackage{setspace}% pour les interlignes
%\onehalfspacing %interligne 1.5
%\doublespacing %interligne 2
%\renewcommand{\baselinestretch}{1.5}  %interligne défini
%\usepackage{vmargin}% pour les marges
%\setmarginsrb{2.5}{2.5}{2.5}{2.5}{}{}{}{} % marges de 2.5 cm 
%\addto\captionsfrench{\def\tablename{Tableau}} % pour avoir TABLEAU et pas TABLE dans la légende des tableaux..
%\setlength{\parskip}{1cm}   %espacement fixe entre chaque paragraphe
\setlength{\parindent}{1cm}  %modifie la valeur de l'alinéas
%\addtolength{\voffset}{-1.5cm} % (diminue la marge du haut)
\addtolength{\textheight}{-2cm} % (augmente la longueur du texte)
%\addtolength{\hoffset}{-1cm} (diminue la marge de gauche)
%\addtolength{\textwidth}{2cm}  (augmente la largeur du texte)
%\addtocounter{secnumdepth}{1}  si jamais on veut utiliser \subsubsubsecion
\usepackage[hang,center,bf]{caption} %pour les légendes
\setlength{\captionmargin}{30pt}
\usepackage[hang,flushmargin]{footmisc} %à mettre avec ENGLISH dans babel pour avoir les notes de bas de page à gauche et non indentées
\usepackage[nonumberlist,style=altlist,toc]{glossaries} % Pour faire un glossaire
\makeglossaries
%\addto\captionsfrench{\renewcommand*{\glossaryname}{Glossary}}
\usepackage{wasysym}
\usepackage[square, numbers, comma, sort&compress]{natbib} % Use the natbib reference package - read up on this to edit the reference style; if you want text (e.g. Smith et al., 2012) for the in-text references (instead of numbers), remove 'numbers' 
%\hypersetup{urlcolor=blue, colorlinks=true} % Colors hyperlinks in blue - change to black if annoying
\title{Project 2 - Constraint Programming } % Defines the thesis title - don't touch this
%-------------------------------------------------------------------------------------------------------------------------------------------------------------




\begin{document}

\definecolor{dkgreen}{rgb}{0,0.6,0}
\definecolor{gray}{rgb}{0.5,0.5,0.5}
\definecolor{mauve}{rgb}{0.58,0,0.82}

\lstset{ %
  language=c,                				% the language of the code
  basicstyle=\footnotesize,           	% the size of the fonts that are used for the code
  numbers=left,                   			% where to put the line-numbers
  numberstyle=\tiny\color{gray},  	% the style that is used for the line-numbers
  stepnumber=1,                   			% the step between two line-numbers. If it's 1, each line 
                                  						% will be numbered
  numbersep=5pt,                  			% how far the line-numbers are from the code
  backgroundcolor=\color{white},   % choose the background color. You must add \usepackage{color}
  showspaces=false,               % show spaces adding particular underscores
  showstringspaces=false,         % underline spaces within strings
  showtabs=false,                 % show tabs within strings adding particular underscores
  frame=single,                   % adds a frame around the code
  rulecolor=\color{black},        % if not set, the frame-color may be changed on line-breaks within not-black text (e.g. commens (green here))
  tabsize=4,                      % sets default tabsize to 2 spaces
  captionpos=b,                   % sets the caption-position to bottom
  breaklines=true,                % sets automatic line breaking
  breakatwhitespace=false,        % sets if automatic breaks should only happen at whitespace
  title=\lstname,                   % show the filename of files included with \lstinputlisting;
                                  % also try caption instead of title
  keywordstyle=\color{blue},          % keyword style
  commentstyle=\color{dkgreen},       % comment style
  stringstyle=\color{mauve},         % string literal style
  escapeinside={\%*}{*)},            % if you want to add LaTeX within your code
  morekeywords={*,...}               % if you want to add more keywords to the set
}

\floatstyle{plain}
%\newfloat{graphique}{!hb}{lgr}[chapter]
\floatname{graphique}{Graph}

%\setstretch{1.1} % Line spacing of 1.3

% Define the page headers using the FancyHdr package and set up for one-sided printing
\fancyhead{} % Clears all page headers and footers
\rhead{\thepage} % Sets the right side header to show the page number
\lhead{} % Clears the left side page header

\pagestyle{fancy} % Finally, use the "fancy" page style to implement the FancyHdr headers

\newcommand{\HRule}{\rule{\linewidth}{0.5mm}} % New command to make the lines in the title page

\begin{titlepage}
\pagestyle{fancy} % Finally, use the "fancy" page style to implement the FancyHdr headers

\begin{tabular}{cc}
\begin{minipage}{0.5\textwidth}
\begin{flushleft}
\includegraphics[scale=0.1]{./logoingisbleu.jpg} % University/department logo - uncomment to place it
\end{flushleft}
\end{minipage}
 & 
 \begin{minipage}{0.43\textwidth}
\begin{flushright}
\includegraphics[scale=0.5]{./epl.jpg} % University/department logo - uncomment to place it
\end{flushright}
\end{minipage}
\end{tabular} 



\begin{center}
\vspace{100 px}
\textsc{\LARGE Catholic University of Louvain}\\[1cm] % University name
\textsc{\Large Project 4 : Modeling}\\[0.5cm] % Thesis type
 
\HRule \\[0.4cm] % Horizontal line
{\huge \bfseries LINGI2365 - Constraint Programming}\\[0.4cm] % Thesis title
\HRule \\[1.5cm] % Horizontal line
 

\begin{tabular}{cc}
\begin{minipage}{0.5\textwidth}
\begin{flushleft} \large
\emph{Auteurs:}\\
{Vanwelde Romain (3143-10-00)\\
Crochelet Martin (2236-10-00)\\ \ \\
Groupe 7} 
\end{flushleft}
\end{minipage} & \begin{minipage}{0.46\textwidth}
\centering
\begin{flushright} \large
\emph{Superviseurs:} \\
{Pr. Yves Deville\\
François Aubry
}
\end{flushright}
\end{minipage}\\[3cm] \\ 
\end{tabular} 

 
%\large \textit{A thesis submitted in fulfilment of the requirements\\ for the degree of \degreename}\\[0.3cm] % University requirement text
%\textit{in the}\\[0.4cm]
%\groupname\\\deptname\\[2cm] % Research group name and department name

 \begin{center}
{\large \today }\\[4cm] % Date 
 \end{center}


\vfill
\end{center}

\end{titlepage}

\lhead{\emph{LINGI2365 - Constraint Programming}} % Set the left side page header to "Contents"
\tableofcontents % Write out the Table of Contents

\thispagestyle{fancy}

\pagebreak
\setcounter{page}{1}
\pagestyle{fancy} % Finally, use the "fancy" page style to implement the FancyHdr headers

\section{Louvain-La-Neuve Golfer Problem}
\subsection{Explain which symmetries can arise for this problem.}
\subsection{Describe two possible models for this problem. }
\subsubsection{explain your models in detail}
\subsubsection{explain how you can modify your models to take symmetries into account}
\subsection{Implement both models (considering symmetries) in Comet.}
\subsection{Explain which variable / value ordering heuristics you use with each model.}
\subsection{What is the theoretical maximum number of week in a schedule?}
\subsection{Indicate the maximum number of weeks you could identify in a reasonable time limit using both models.}
\subsection{Indicate for each model, for each number of weeks the time needed to find a solution, the number of failures and the number of choices. Explain the results, do they correspond to what you would have expected?}

\section{The Time Tabling Problem}
\subsection{Explain which symmetries arise in this problem.}
\subsection{Design an efficient model for this problem.}
\subsection{Explain your model for this problem and explain how it handles symmetries.}
\subsection{Implement your model in Comet.}
\subsection{To solve this problem you will need a search procedure that is more efficient than a simple label.}
\subsection{Test your model on each of the instances provided on the iCampus site. Indicate for each instance the time needed to solve it, the number of failures and the number of choices.}

\end{document}